\documentclass[12pt]{article}

\usepackage[T1]{fontenc}
\usepackage{enumitem}
\usepackage[ampersand]{easylist}
\usepackage{amsfonts} %need for set symbols
\usepackage{mathtools}
\usepackage{hyperref}

\usepackage{ThomasNotes}

\title{}
\author{Thomas Jakway}
\date{}
\begin{document}
\maketitle

\section*{Glossary}
\begin{enumerate}
    \item $p$ = momentum, as in $p = mv$ (momentum = mass times velocity).  Vector quantity.
\end{enumerate}

\section*{2018-05-25 Notes}
g(a): $m/s^2$ [acceleration] = gravity at altitude

resultant force = thrust - weight
acceleration = resultant force (Newtons) / mass (kilograms)

as fuel is burned the rocket becomes lighter


p(t) = position as a function of time
p(t)

acceleration(t) = acceleration as a function of time
\[acceleration(t) = rocket thrust - gravity \]

\^{} need to add a function for propellant decreasing (\emph{mass ratio}):
    

\[speed(t) = a(t) + speed(t - 1) \]

speed is a fold over speed (just like position is a fold over position + speed)
but acceleration doesn't contain reference to previous acceleration

%**************************************************************
\section*{2018-06-15 Notes}

\begin{notetaking}

From \url{http://hyperphysics.phy-astr.gsu.edu/hbase/rocket.html#c2}:
\begin{quote}
The net external force acting on an object can be evaluated as the rate of change of momentum. This turns out to be a more fundamental way of stating the force than the use of Newton's second law.

\emph{The average force on a constant mass system is seen to be equal to the rate of change of momentum.} [Emphasis mine]
\end{quote}

Generalization that allows for changing mass:



\end{document}
